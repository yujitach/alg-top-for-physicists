\documentclass[12pt]{article}


\usepackage{amsmath}
%\usepackage{amsthm}
\usepackage{amssymb}

\usepackage[nottoc]{tocbibind}
\usepackage[usenames,dvipsnames,svgnames,table]{xcolor}
\usepackage[colorlinks,citecolor=DarkGreen,linkcolor=FireBrick,urlcolor=FireBrick,linktocpage]{hyperref}
\numberwithin{equation}{section}
\urlstyle{rm}
\def\arxivfont{\rm}
\def\doihref#1#2{\href{#1}{{\color{FireBrick!80!Black}#2}}}
\usepackage{graphicx}
%\usepackage{showkeys}
\usepackage{float}
\usepackage[height=8.8in,width=6.45in]{geometry}
\usepackage{tgtermes}
\usepackage{tgadventor}
\usepackage[scaled]{couriers}
\usepackage[T1]{fontenc}
\usepackage[mathscr]{euscript}
\usepackage[export]{adjustbox}
\usepackage{braket}
\usepackage[safe]{tipa}

\usepackage{xcolor}
\usepackage{mdframed}
\usepackage[amsmath]{ntheorem}
%\theorembodyfont{\normalfont}

\newmdtheoremenv[backgroundcolor=black!10,linecolor=black!0]{definition}{Definition}[section]
\newmdtheoremenv[backgroundcolor=black!10,linecolor=black!0]{theorem}[definition]{Theorem}
\newmdtheoremenv[backgroundcolor=black!3,linecolor=black!0]{example}[definition]{Example}
\newmdtheoremenv[backgroundcolor=black!3,linecolor=black!0]{notation}[definition]{Notation}
\newmdtheoremenv[backgroundcolor=black!0]{exercise}[subsection]{Exercise}



\newenvironment{claim}{  \begin{mdframed}[linecolor=black!0,backgroundcolor=black!10]\noindent\ignorespaces}{\end{mdframed}}

\let\originalfigure=\figure
\let\endoriginalfigure=\endfigure

\renewenvironment{figure}[1][]{
  \begin{originalfigure}[#1]
    \begin{mdframed}[linecolor=black!0,backgroundcolor=black!1]
}{
    \end{mdframed}
  \end{originalfigure}
}


\let\originaltable=\table
\let\endoriginaltable=\endtable

\renewenvironment{table}[1][]{
  \begin{originaltable}[#1]
    \begin{mdframed}[linecolor=black!0,backgroundcolor=black!1]
}{
    \end{mdframed}
  \end{originaltable}
}


\def\baselinestretch{1.04}
% https://tex.stackexchange.com/questions/473627/set-the-short-display-skip-as-in-coxeter-regular-complex-polytopes-1974
%\makeatletter
%\g@addto@macro\normalsize{%
%  \setlength\abovedisplayshortskip{\glueexpr\abovedisplayskip-\baselineskip}%
%  \setlength\belowdisplayshortskip{\belowdisplayskip}%
%}
%\makeatother

\def\Nequals#1{$\mathcal{N}{=}\,#1$}


\definecolor{shadecolor}{rgb}{0.90,0.90,0.90}


\def\important#1#2{%
\begin{mdframed}[linecolor=black!0,backgroundcolor=black!5]
\textbf{#1.} #2
\end{mdframed}
}



\newenvironment{comment}{\par\noindent [Comment. \ }{\par\noindent End comment.]}

\def\inc#1{\vcenter{\hbox{\includegraphics[scale=.2]{#1}}}}
\def\incc#1{\vcenter{\hbox{\includegraphics[scale=.6]{#1}}}}

\def\TODO#1{{\color{FireBrick} #1}}

\usepackage[whole]{bxcjkjatype} 

\definecolor{identifiercolor}{rgb}{.4,.6,.56}
\definecolor{stringcolor}{gray}{0.5}
\definecolor{inactivecolor}{rgb}{0.15,0.15,0.5}

%%list alg-top-notes

\def\bC{\mathbb{C}}
\def\bH{\mathbb{H}}
\def\bR{\mathbb{R}}
\def\bZ{\mathbb{Z}}

\begin{document}

\centerline{\Large Algebraic topology for physicists}

\bigskip

\centerline{\large Yuji Tachikawa}

\setcounter{tocdepth}{2}
\tableofcontents

\newpage



\section{General introduction}

\subsection{Why now?}
\label{sec:whynow}

Even though there are a number of mathematical physicists in our physics department,
there has been no course on mathematical physics in the graduate school,
at least since when I got hired about ten years ago.
I always wanted to give one such course, but I know 
there are a lot of bureaucratic hurdles to be cleared before adding a lecture slot with a new subject name in the curriculum.

Last year, I noticed that there already actually is a lecture slot with the name
数理物理学 (mathematical physics)
which was somehow not used for about 20 years.
It turns out reviving a long dormant lecture slot requires almost no paperwork,
so I decided to do just that.

Mathematical physics can mean many things. 
It is often distinguished from mathematical methods for physicists (which has a distinct translation in Japanese, 物理数学).
It often means those parts of theoretical physics where the discussions are mathematically rigorous,
e.g.~the part of statistical physics where the existence of thermodynamic limit or of phase transitions are rigorously proved.
It can also mean those parts of theoretical physics where various mathematical concepts are heavily used, albeit not quite rigorously,
such as string theory.


I'm not sure how other faculty member would use this slot in the future,
but my goal this year is to provide a introduction to algebraic topology for physicists,
so it's closer to a course on mathematical methods for physicists,
although the sub-subject of mathematics covered is somewhat different from the usual ones (calculus, linear algbera, a bit of group theory, etc.).
My rationale is the following.

As you already know, math plays a very important role in physics, 
as was famously pointed out by Wigner in his essay \cite{WignerUnreasonable}.
But it is definitely \emph{not} that all subfields of math are equally important.
Clearly important ones are:
\begin{itemize}
  \item \textbf{Calculus:} in some sense this subject arose from physics (by Newton etc.)
  \item \textbf{Linear algebra:} this is important for anything which deals with first-order approximations. 
  It is also a crucial ingredient of quantum mechanics (QM).
  \item \textbf{Theoy of Hilbert spaces:} equally important for QM.
  \item \textbf{Group theory:} Symmetry is one of the fundamental concepts in physics. It gives rises to conserved charges in both analytical mechanics and quantum mechanics, for example.
  \item \textbf{Differential geometry:} this is important for general relativity (GR).
\end{itemize}

But there are also ones whose usefulness is quite dubious (or at least not immediately obvious):
\begin{itemize}
  \item \textbf{Number theory:} physics is primarily based on $\bR$ and $\bC$, 
  while the number theory is about $\bZ$.
  \item \textbf{Algebraic geometry:} this is about shapes of objects defined by polynomial equations. 
  Why should physicists care about polynomial equations?
  \item \textbf{Mathematical logic / set theory:} Will we ever use G\"odel's incompleteness theorem in physics?
  Will physics depend on the choice of the particular axioms of set theory?
\end{itemize}

\textbf{Algebraic topology} was in the middle of these two lists, until about 15 years ago.
From 1970s, basic homotopy theory was used in the study of topological solitons 
in particle physics and in condensed matter physics.
Starting in the 1980s, 
some characteristic classes were used in the study of anomalies in particle physics
and in the study of quantum Hall effects in condensed matter physics.
But that was about it.
The algebraic topology used was also quite elementary from mathematician's perspective:
everything used on the physics side was developed 1940s, say.
So  algebraic topology is not usually (and does not have to be) covered in a typical physics curriculum,
or textbooks on mathematical methods for physicists.

But that changed in the last 15 years. 
Topological insulators and superconductors became a hot topic,
and soon we learned that they are classified by K and KO theory \cite{Schnyder:2008tya,Kitaev:2009mg,Ryu:2010zza}.
And they are classification of non-interacting phases.
The study of interacting phases, of a class often known as \emph{symmetry protected topological (SPT) phases},
or as \emph{invertible phases}, 
pursued e.g.~in \cite{Fidkowski:2009dba,Chen:2011pg,Gu:2012ib,Metlitski:2014xqa} 
soon gave rise to the realization that the classification would be done by bordism groups \cite{Kapustin:2014dxa}
or by more general cohomology theories \cite{KitaevCollapse}.
This expectation was confirmed later by \cite{Freed:2016rqq,Yonekura:2018ufj},
where it was shown that the classification of the SPT phases is done by 
the Anderson (or Pontryagin) dual of the bordism group.
This also implicitly gave a general theory of anomalies on the particle physics side.
To understand all this requires a far more advanced algebraic topology than was necessary before
(although still not very modern from mathematicians' point of view, 
since it only uses algebraic toplogy up to 1970 or something).

To understand this fascinating development required me to learn bits and pieces of algebraic topology 
from various sources. But there is no single place where most of the relevant materials for physicists
are gathered. 
This set of lectures is my attempt to provide such a place.

\subsection{What is algebraic topology? Is it any good?}

Topology is a subfield of math where people study spaces.
Spaces of course appear in physics. 
The four-dimensional spacetime we live in and study via general relativity is a mathematical space.
The space of configurations of a rigid body is also a space.
Similarly, the order parameter of a condensed matter system is a space.
The space of all possible gapped condensed-matter Hamiltonians defined on a lattice 
is also a space.
So, methods to study spaces (i.e.~topology) should be useful to us.

In algebraic topology, we study spaces by associating algebraic objects to them.
Given a space $X$, some of the algebraic objects associated are:
\begin{itemize}
\item \textbf{Homotopy groups:} $\pi_n(X)$ is the group obtained by considering maps $S^n \to X$,
where $S^n$ is the $n$-dimensional sphere (in the convention that the ordinary sphere is $S^2$).
\item \textbf{Bordism groups:} $\Omega_n(X)$ is the group obtained by considering maps $M_n \to X$,
where $M_n$ is a general smooth manifold of dimension $n$.
\item \textbf{Homology groups:} $H_n(X)$ is the group obtained by considering $n$-dimensional `cycles' (essentially polyhedra)
 in $X$, where a cycle is a more general notion than spheres or manifolds.
\end{itemize}
Note that the `source spaces' in the descriptions above has the inclusion relation \begin{equation}
  \{\text{spheres}\} \subset \{\text{manifolds}\} \subset \{\text{cycles}\}
\end{equation} and somehow the difficulty of the computation decreases as the source spaces become more general,
so that the homotopy groups are the most difficult to compute and the homology groups are the easiest.

Homotopy groups are useful to describe topological solitons,
and bordism groups play important roles in the study of topological phases
and of anomalies of quantum field theories.
We will also encounter K-theory $K(X)$, whose difficulty of computation and of definition lies between homotopy and homotopy 
and is similar to that of bordism groups, 
which is useful in the study of non-interacting topological phases
and of properties of fermion fields in quantum field theories.
Homology groups are not as directly relevant for physics at this point of our discussion,
but as the easiest to compute, obtaining them is a first step in the computation of other algebraic objects.

Another topic is the study of fiber bundles $E$ over a space $X$.
This is a space $E$ with a projection $E\to X$ such that, locally around each point $x\in X$, 
$E$ looks like $U\times F$ where $x\in U\subset X$ is a neighborhood of $X$ and $F$ is a fixed space called the fiber.
Topology of fiber bundles are often distinguished by the help of characteristic classes,
again a topic of algebraic topology.
This appears in many places in physics. Three examples:
\begin{itemize}
\item Gauge fields (such as the Maxwell field or the Yang-Mills field, describing the electromagnetism or the strong force)
mare mathematically described by connections on fiber bundles, where the base space $X$ is our spacetime.
The topology of this fiber bundle gives rise to Dirac monopoles and instantons,
specified by the characteristic classes known as the 1st and 2nd Chern classes of the bundle, respectively.
\item Consider a crystalline system in a condensed matter setting. The band structure of the system
determines a fiber bundle (of Hilbert spaces) over the Brillouin zone. 
Its topology underlie many of the topological properties of the system.
For example, for a $2+1$ dimensional system, 
the 1st Chern class of this bundle (which is an integer) gives the quantized Hall conductance,
as the famous analysis of Thouless, Kohmoto, Nightingale, and den Nijs showed \cite{Thouless:1982zz}.
\item A parameterized family of quantum states can also be considered as a fiber bundle over the parameter space,
where the fiber can be either the Hilbert space of wavefunctions or the space of density matrices.
The topology of this fiber bundle gives rise to the notion of Berry phase.
We will also see that there are cases such that there can be a continuous family of density matrices
which is never realized as a continuous family of statistical mixture of wavefunctions.
Again this issue is detected by a characteristic class, known as the Dixmier-Douady class.
\end{itemize}
Characteristic classes of fiber bundles often take values in cohomology groups of the base space $X$,
which also explains the usefulness of (co)homology groups in physics.

The aim of this lecture series is to provide a minimal amount of information so that you can understand 
the content of the paragraphs above.
This will, unfortunatley, require the whole semester.

\subsection{Mathematical rigor in the course}
This is not a course for (prospective) mathematicians. 
For me, math is like a collection of useful apps (on your Mac/PC or mobile devices).
Here is a comparison chart:
\begin{center}
\begin{tabular}{|c|c|}
  \hline
  math & app \\
  \hline
  definition & short usage of the app \\
  theorem & app itself \\
  proof & source code of the app \\
  \hline
\end{tabular}
\end{center}
Reading the source code of an open source app can be fun and instructive,
but not necessarily required if you only want to use the app. 
Similarly, even reading the short usage of the app (say in an app store) can be too much 
if you want to use the app just once, for example.
In this course, we will not going into the source code of the apps,
but I do intend tell you what kind of apps are available
and give at least short usages of the apps involved.

I should say that papers and textbooks in theoretical physics are not as neatly packaged as those in math.
In math, you can skip the proofs but still use the theorems to compute what we need.
In contract, in papers on physics, we freely go back and forth between assumptions, derivations and conclusions,
and we are almost always required to read the whole paper to understand what is going on and to use it for our own purposes.
Not only that, in math, the proofs are usually reliable, so you can just use the theorems without worrying too much.
In physics, the statements often relies on unwritten assumptions, so we need utmost caution when we want 
to use results in other papers. 
Maybe we physicists have something to learn from mathematicians in this regard...


\subsection{Aside: some rare use of `useless' math subfields in physics}
Before proceeding to the main part of the lectures,
I'd like to mention some meager connections to physics of `useless' subfields of math I mentioned in Sec.~\ref{sec:whynow}.

\paragraph{Number theory:}
Modular functions appear and are used in the study of two-dimensional conformal field theory (CFT) and in string theory.
Modular functions also play very important roles in number theory.
For example there is an introductory book on number theory by Mieda \cite{Mieda} published this year,
where exactly the same functions I often see in string theory and in two-dimensional CFT appears throughout the book.
Whether string theory is physics is debatable, but I think 2d CFT definitely is.
So there is at least a small connection between number theory and physics.
Is it an accident? Or is there a deeper relation?
I should say that there is even a journal called \emph{Number Theory and Physics}.

\paragraph{Algebraic geometry:}
In algebraic geometry people study the shapes of spaces determined by polynomial equations. 
In physics the shapes of spaces are usually determined by differential equations.
Is there a place where polynomial equations naturally appear in physics?
If you are kind enough to consider string theory physics, then the answer is yes.
In string theory the spacetime is ten dimensional.
To describe our four-dimensional world, we need to assume that the extra $10-4=6$ dimensions are compactified.
The real world is not supersymmetric, but if we assume supersymmetry as a way to acquaint ourselves with the sytem,
then the compactified part of the space is described by a six-dimensional Calabi-Yau manifold.
There is a mathematical theorem saying that Calabi-Yau manifolds are always described by polynomial equations.

\paragraph{Mathematical logic / set theory:}
In condensed matter systems, a standard toy model is a spin chain:
we have sites labeled by an integer $i=-3, -2, -1, 0, 1, 2, 3, \ldots$
and at each site we have a qubit $\ket{\pm}_i$. 
We consider a translation-invariant Hamiltonian consisting of local interactions,
such as the  standard Ising model: \begin{equation}
H = \sum_i [ a (\sigma_X)_i + b (\sigma_Z)_i (\sigma_Z)_{i+1} ].
\end{equation}
This class of systems is known to exhibit various interesting phenomena,
and countless person-hours have been spent in understanding it. 

One basic question is whether the ground state is gapped or not. 
In the case of the Ising model above, it is gapped when $|a|\neq |b|$ and it is gapless (and flows to a conformal field theory) when $|a|=|b|$.
One can ask: can there be a computer algorithm
which determines whether a given local, translationally-invariant Hamiltonian leads to a gapped ground state or not?
Clearly theoretical physicists are not clever enough to do this,
but can we imagine a day in the far future where such a thing is possible?

The answer is no \cite{2dNature,2dLong,1d}.
The point is that, by a very clever construction explained in the references cited,
one can write down a Hamiltonian for any computer program (or more precisely a Turing machine)
such that the ground state on a finite chain of $L$ sites is gapped 
if the computer program stops in $L$ steps,
and not gapped otherwise.
Therefore, if there is an algorithm which can decide whether a given Hamiltonian leads to a gapped ground state on an infinite chain,
the same algorithm can determine whether a given Turing machine stops or not.
But a very basic fundamental theorem of computational science says that
there is no such algorithm determining whether a given Turing machine stops or not.
So this is impossible.

Similarly, it is easy to write down a program which stops if and only if the current standard axioms of set theory is inconsistent: 
we simply enumerate all possible sequences of alphabets.
Many of them are garbage, but some of them describes a proof of a mathematical theorem,
and any possible proof of any mathematical theorem eventually appears along the enumeration process.
We let the program stops if it finds a proof of $0=1$, which happens if and only if the current standard axioms of set theory is inconsistent. 
We can encode this program as a spin chain Hamiltonian.
Then this Hamiltonian has a gapped ground state if and only if the current standard axioms of set theory is inconsistent. 
Then G\"odel's second incompleteness theorem says that one can neither prove nor disprove that this Hamiltonian leads to a gapped ground state or not.

There are other works of similar nature inspired by this work.
For example, in \cite{thermalNature,thermalLong}, it was shown that 
there is no algorithm which tells a given quantum state thermalizes or not.
I also constructed an example in supersymmetric quantum field theory (SQFT):
there can not be an algorithm which tells whether a given 2d SQFT has a supersymmetric vacuum or not \cite{Tachikawa:2022vsh}.

All this was quite fascinating to me, 
but clearly this is not a very deep application of mathematical logic / set theory / computational science.
This is because the main mathematical facts used (G\"odel's incompleteness theorems, or the undecidability of the halting problem of Turing machines) are both fundamental but very old.


\section{Manifolds}

We have two intuitions about spaces.
One is such that each point is locally like $\bR^n$.
Another is such that it is made up from pasting small triangles (for surfaces) or tetrahedra (for 3-dimensional objects) and analogous constructions in higher dimensions.
We use both ideas to study spaces. 
Let's start with some definitions.

\subsection{Definition of manifolds}

\begin{definition}
A \emph{(topological) manifold} $M$ of dimension $n$ is such that for each point $p\in M$ there is a neighborhood $U\subset M$ of $p$ and a neighborhood $U'\subset \bR^n$ of $0$ such that 
there is a bijective \emph{continuous} map $f:U\to U'$.
\end{definition}

\begin{definition}
  A \emph{smooth manifold} $M$ of dimension $n$ is such that for each point $p\in M$ there is a neighborhood $U\subset M$ of $p$ and a neighborhood $U'\subset \bR^n$ of $0$ such that 
  there is a bijective \emph{smooth} map $f:U\to U'$.  
\end{definition}

In physics such $U$'s are often called coordinate patches. 
We only consider manifolds `good enough' such that 
there is a set of coordinate patches $U_i$ covering the whole manifold $M$,
such that for each point $p\in M$ there is only a finite number of coordinate patches containing $p$.
In such cases we can think of manifolds as being built from
pasting together the coordinate patches $U_i \subset \bR^n$
by the maps \begin{equation}
f_{ij}: U_i\cap U_j \to U_i\cap U_j
\end{equation} whenever the intersection $U_i\cap U_j$ is non-empty.

A smooth manifold is such that $f_{ij}$ are smooth;
a topological manifold is such that $f_{ij}$ are just required to be continuous.
A smooth manifold is automatically a topological manifold.
A bijective smooth map $f:M\to M'$ between two smooth manifolds $M$ and $M'$
is called a \emph{diffeomorphism};
a bijective continuous map $f:M\to M'$ between two topological manifolds $M$ and $M'$
is called a \emph{homeomorphism}.
In physics we typically consider smooth manifolds,
but I decided to include a bit of discussions about topological manifolds
as I thought the subtle concrete differences between smooth manifolds
and topological manifolds can be interesting to some of you.

\subsection{Examples}

\subsubsection{Via equations}
A common way to construct manifolds is to take a subset of $\bR^n$ defined by a number of equations:
\begin{align}
f_1(x_1, x_2, \ldots, x_n) &= 0, \\
f_2(x_1, x_2, \ldots, x_n) &= 0, \\
&\vdots \\
f_m(x_1, x_2, \ldots, x_n) &= 0.
\end{align}
If the choice of $f_{1,2,\ldots,m}$ are sufficiently nice,
the set of points satisfying these equations is a manifold of dimension $n-m$.
A typical bad example is the subset of $\bR^2$ defined by $xy=0$: around $(x,y)=(0,0)$, two lines intersect, and therefore the neighborhood of $(0,0)$ is not of the right form.

\begin{example}
The $n$-dimensional sphere $S^n$ is defined by \begin{equation}
  x_1^2+x_2^2+\cdots+x_{n+1}^2=1.
\end{equation} 
\end{example}
Note that $S^n$ is embedded in $\bR^{n+1}$, but it is a manifold of dimension $n$.

We have various group manifolds, 
defined in a similar manner. 
Let $M_n(\bR)$ be the set of $n\times n$ real matrices.
\begin{example}
The orthogonal group $O(n)$ is defined by \begin{equation}
  O(n) = \{ M\in M_n(\bR) \mid M^T M = 1 \}.
\end{equation}
The special orthogonal group $SO(n)$ is defined by \begin{equation}
  SO(n) = \{ M\in O(n) \mid \det M = 1 \}.
\end{equation}
\end{example}

Similarly let $M_n(\bC)$ be the set of $n\times n$ complex matrices.
\begin{example}
The unitary group $U(n)$ is defined by \begin{equation}
  U(n) = \{ M\in M_n(\bC) \mid M^\dagger M = 1 \}.
\end{equation}
The special unitary group $SU(n)$ is defined by \begin{equation}
  SU(n) = \{ M\in U(n) \mid \det M = 1 \}.
\end{equation}
\end{example}

I don't know about you but I like quaternions $\bH$.
This is a four-dimensional algebra over $\bR$ with a basis $1,i,j,k$ satisfying \begin{equation}
  i^2=j^2=k^2=-1, \quad
  ij=-ji=k, \quad
  jk=-kj=i, \quad
  ki=-ik=j.
\end{equation}
A general element is of the form $q=a+bi+cj+dk$ with $a,b,c,d\in \bR$.
The conjugate of $q=a+bi+cj+dk$ is $\bar q=a-bi-cj-dk$.
The norm of $q$ is $|q|^2=q\bar q=\bar q=a^2+b^2+c^2+d^2$.
This is nonzero unless $q=0$.
This means that ...

\subsubsection{Via combinations}


\begin{notation}
 Given two manifolds $M$ and $N$ of dimensions $m$ and $n$,
 we write its product as $M\times N$.
It is a manifold of dimension $m+n$.
\end{notation}

\begin{notation}
Given two manifolds $M$ and $N$ of the same dimension $n$,
we write its disjoint union as $M\sqcup N$.
It is a manifold of dimension $n$.
\end{notation}

Note that $S^1\times S^1$ is two-dimensional and the surface of a donut,
while $S^1\sqcup S^1$ consists of two circles and is one-dimensional.

\begin{example}
  The $n$-dimensional torus is \begin{equation}
    T^n = \underbrace{S^1\times S^1\times \cdots \times S^1}_\text{ $n$ times}.
  \end{equation}
\end{example}

\subsubsection{Via quotients}


Another method to define manifolds is to 
take the quotient of a manifold by a group action. 
Let us explain it more fully.
\begin{definition}
A \emph{group action} of a group $G$ on a manifold $M$ is a map \begin{equation}
  G\times M \to M, \quad (g, p) \mapsto g p
\end{equation} such that \begin{equation}
  e p = p, \quad (gh) p = g (h p)
\end{equation} for all $g,h\in G$ and $p\in M$.
Here $e$ is the identity in $G$.
\end{definition}
We often want the action to be smooth or at least continuous,
depending on the context.

\begin{definition}
Let $p\sim q$ if there is a $g\in G$ such that $g p = q$.
The quotient space $M/G$ is the set $M/\sim$ of equivalence classes 
under this relation.
\end{definition}

$M/G$ is not always a manifold.
For example, consider $\bR^3$ with the action of $\bZ_2$ generated by 
\begin{equation}
(x,y,z) \mapsto (-x,-y,-z).
\end{equation}
The origin is singular, and the quotient space $\bR^3/\bZ_2$ is not a manifold.
(It still belongs to a larger class of spaces called \emph{orbifolds}.)

\begin{theorem}
  If  $g\in G$ other than the identity $e$
  satisfy $p\neq gp$ for all $p\in M$,
  then $M/G$ is a manifold.
\end{theorem}

\section{Fiber bundles}

\section{Basic homotopy theory}

\section{Differential forms and de Rham cohomology}

\section{Cellular cohomology}

\section{Obstruction classes}

\section{Classifying spaces}

\section{Chern-Simons terms and bordism groups}

\section{K-theory}

\section{Atiyah-Singer index theorem}

\section{Anomalies}

\section{Computing \texorpdfstring{$\pi_4(S^3)$}{pi4(S3)}}

\bibliographystyle{ytamsalpha}
%\baselineskip=.85\baselineskip%\let\ttfamily\relax
%\let\bbb\bibitem\def\bibitem{\itemsep1pt\bbb}
\bibliography{ref}

\end{document}

.