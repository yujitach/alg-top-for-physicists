\documentclass[12pt]{article}

\usepackage{amsmath}
\usepackage{amsthm}
\usepackage{amssymb}

\usepackage[nottoc]{tocbibind}
\usepackage[usenames,dvipsnames,svgnames,table]{xcolor}
\usepackage[colorlinks,citecolor=DarkGreen,linkcolor=FireBrick,urlcolor=FireBrick,linktocpage]{hyperref}
\numberwithin{equation}{section}
\urlstyle{rm}
\def\arxivfont{\rm}
\def\doihref#1#2{\href{#1}{{\color{FireBrick!80!Black}#2}}}
\usepackage{graphicx}
%\usepackage{showkeys}
\usepackage{float}
\usepackage[height=8.8in,width=6.45in]{geometry}
\usepackage{tgtermes}
\usepackage{tgadventor}
\usepackage[scaled]{couriers}
\usepackage[T1]{fontenc}
\usepackage[mathscr]{euscript}
\usepackage[export]{adjustbox}
\usepackage{braket}
\usepackage[safe]{tipa}

\usepackage{xcolor}
\usepackage{mdframed}
\newenvironment{claim}{  \begin{mdframed}[linecolor=black!0,backgroundcolor=black!10]\noindent\ignorespaces}{\end{mdframed}}

\let\originalfigure=\figure
\let\endoriginalfigure=\endfigure

\renewenvironment{figure}[1][]{
  \begin{originalfigure}[#1]
    \begin{mdframed}[linecolor=black!0,backgroundcolor=black!1]
}{
    \end{mdframed}
  \end{originalfigure}
}


\let\originaltable=\table
\let\endoriginaltable=\endtable

\renewenvironment{table}[1][]{
  \begin{originaltable}[#1]
    \begin{mdframed}[linecolor=black!0,backgroundcolor=black!1]
}{
    \end{mdframed}
  \end{originaltable}
}


\def\baselinestretch{1.04}
% https://tex.stackexchange.com/questions/473627/set-the-short-display-skip-as-in-coxeter-regular-complex-polytopes-1974
%\makeatletter
%\g@addto@macro\normalsize{%
%  \setlength\abovedisplayshortskip{\glueexpr\abovedisplayskip-\baselineskip}%
%  \setlength\belowdisplayshortskip{\belowdisplayskip}%
%}
%\makeatother

\def\Nequals#1{$\mathcal{N}{=}\,#1$}

\usepackage{framed}
\definecolor{shadecolor}{rgb}{0.90,0.90,0.90}
\theoremstyle{definition}
\newtheorem{exercise_}{Exercise}[section]
\newenvironment{exercise__}%
{\begin{exercise_}\begin{shaded}}%
{\end{shaded}\end{exercise_}}
\def\exercise#1{\begin{exercise__}
#1
\end{exercise__}}

\def\answer{\paragraph{Answer.} }


\def\important#1#2{%
\begin{mdframed}[linecolor=black!0,backgroundcolor=black!5]
\textbf{#1.} #2
\end{mdframed}
}


\newenvironment{comment}{\par\noindent [Comment. \ }{\par\noindent End comment.]}

\def\inc#1{\vcenter{\hbox{\includegraphics[scale=.2]{#1}}}}
\def\incc#1{\vcenter{\hbox{\includegraphics[scale=.6]{#1}}}}

\def\TODO#1{{\color{FireBrick} #1}}

\usepackage{bxcjkjatype} 

\definecolor{identifiercolor}{rgb}{.4,.6,.56}
\definecolor{stringcolor}{gray}{0.5}
\definecolor{inactivecolor}{rgb}{0.15,0.15,0.5}

%%list alg-top-notes

\def\bC{\mathbb{C}}
\def\bR{\mathbb{R}}
\def\bZ{\mathbb{Z}}

\begin{document}

\centerline{\Large Algebraic topology for physicists}

\bigskip

\centerline{\large Yuji Tachikawa}

\setcounter{tocdepth}{2}
\tableofcontents

\newpage



\section{General introduction}

\subsection{Motivation of the course}

Even though there are a number of mathematical physicists in our physics department,
there has been no course on mathematical physics in the graduate school,
at least since when I got hired about ten years ago.
I always wanted to give one such course, but I know 
there are a lot of bureaucratic hurdles to be cleared before adding a lecture slot with a new subject name in the curriculum.

Last year, I noticed that there already actually is a lecture slot with the name
\begin{uCJK}数理物理学\end{uCJK} (mathematical physics)
which was somehow not used for about 20 years.
It turns out reviving a long dormant lecture slot requires almost no paperwork,
so I decided to do just that.

I'm not sure how other faculty member would use this slot in the future,
but my goal this year is to provide a introduction to algebraic topology for physicists.
My rationale is the following.

As you already know, math plays a very important role in physics, 
as was famously pointed out by Wigner in his essay \cite{WignerUnreasonable}.
But it is definitely \emph{not} that all subfields of math are equally important.
Clearly important ones are:
\begin{itemize}
  \item Calculus: in some sense this subject arose from physics (by Newton etc.)
  \item Linear algebra: this is important for anything which deals with first-order approximations. 
  It is also a crucial ingredient of quantum mechanics (QM).
  \item Theoy of Hilbert spaces: equally important for QM.
  \item Differential geometry: this is important for general relativity (GR).
\end{itemize}

But there are also ones whose usefulness is quite dubious (or at least not immediately obvious):
\begin{itemize}
  \item Mathematical logic / set theory : Will we ever use G\"odel's incompleteness theorem in physics?
  Will physics depend on the choice of the particular axioms of set theory?
  \item Number theory: physics is primarily based on $\bR$ and $\bC$, 
  while the number theory is about $\bZ$.
  \item Algebraic geometry: this is about shapes of objects defined by polynomial equations. 
  Why should physicists care about polynomial equations?
\end{itemize}

Algebraic topology was in the middle of these two lists, until about 15 years ago.
From 1970s, basic homotopy theory was used in the study of topological solitons 
in particle physics and in condensed matter physics.
Starting in the 1980s, 
some characteristic classes were used in the study of anomalies in particle physics
and in the study of quantum Hall effects in condensed matter physics.
But that was about it.
The algebraic topology used was also quite elementary from mathematician's perspective:
everything used on the physics side was developed 1940s, say.
So  algebraic topology is not usually (and does not have to be) covered in a typical physics curriculum,
or textbooks on mathematical methods for physicists.

But that changed in the last 15 years. 
Topological insulators and superconductors became a hot topic,
and soon we learned that they are classified by K and KO theory \cite{Schnyder:2008tya,Kitaev:2009mg,Ryu:2010zza}.
And they are classification of non-interacting phases.
The study of interacting phases, of a class often known as \emph{symmetry protected topological (SPT) phases},
or as \emph{invertible phases}, 
pursued e.g.~in \cite{Fidkowski:2009dba,Chen:2011pg,Gu:2012ib,Metlitski:2014xqa} 
soon gave rise to the realization that the classification would be done by bordism groups \cite{Kapustin:2014dxa}
or by more general cohomology theories \cite{KitaevCollapse}.
This expectation was confirmed later by \cite{Freed:2016rqq,Yonekura:2018ufj},
where it was shown that the classification of the SPT phases is done by 
the Anderson (or Pontryagin) dual of the bordism group.
This also implicitly gave a general theory of anomalies on the particle physics side.
To understand all this requires a far more advanced algebraic topology than was necessary before
(although still not very modern from mathematicians' point of view, 
since it only uses algebraic toplogy up to 1970 or something).

To understand this fascinating development required me to learn bits and pieces of algebraic topology 
from various sources. But there is no single place where most of the relevant materials for physicists
are gathered. 
This set of lectures is my attempt to provide such a place.

\subsection{Nature of the course}

\subsection{What is algebraic topology, and what is it good for physicists?}

\subsection{Aside: some use of dubious math subfields in physics}

\section{Manifolds}

\section{Fiber bundles}

\section{Basic homotopy theory}

\section{Differential forms and de Rham cohomology}

\section{Cellular cohomology}

\section{Obstruction classes}

\section{Classifying spaces}

\section{Chern-Simons terms and bordism groups}

\section{K-theory}

\section{Atiyah-Singer index theorem}

\section{Anomalies}

\section{Computing \texorpdfstring{$\pi_4(S^3)$}{pi4(S3)}}

\bibliographystyle{ytamsalpha}
%\baselineskip=.85\baselineskip%\let\ttfamily\relax
%\let\bbb\bibitem\def\bibitem{\itemsep1pt\bbb}
\bibliography{ref}

\end{document}

.